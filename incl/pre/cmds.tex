% cmds.tex : custom commands and environments
% ------------------------------------------------------------------------------
% This file contains definitions for custom commands and environments, used for
% shorthand notation for macros used often in the project

% Mathematical symbols ---------------------------------------------------------
\newcommand{\N}{\mathbb{N}}          % natural numbers
\newcommand{\Z}{\mathbb{Z}}          % integers
\newcommand{\Q}{\mathbb{Q}}          % rational numbers
\newcommand{\R}{\mathbb{R}}          % real numbers
\newcommand{\C}{\mathbb{C}}          % complex numbers
\newcommand{\ind}{\mathbbm{1}}       % indicator function
\newcommand{\bigO}{\mathcal{O}}      % big O
\renewcommand{\vec}[1]{\bm{#1}}      % bold vector style

%%
\renewcommand\qedsymbol{$\blacksquare$}                       % tombstone symbol as q.e.d

\newcommand{\Var}{\mathrm{Var}}
\newcommand{\Cov}{\mathrm{Cov}}

\DeclareMathOperator*{\argmin}{arg\,min}
\DeclareMathOperator*{\argmax}{arg\,max}

\DeclarePairedDelimiter\abs{\lvert}{\rvert}                   % absolute value
\newcommand\absscale[1]{\left\lvert#1\right\rvert}            % absolute value with adjusting size
\DeclarePairedDelimiter\norm{\lVert}{\rVert}                  % norm
\newcommand\normscale[1]{\left\lVert#1\right\rVert}           % norm with adjusting size
\DeclarePairedDelimiterX{\inp}[2]{\langle}{\rangle}{#1, #2}   % inner product

\DeclareMathAlphabet{\mathcal}{OMS}{cmsy}{m}{n}               % standard mathcal font

%fix for spacing around adjusted parentheses left
\let\originalleft\left
\let\originalright\right
\renewcommand{\left}{\mathopen{}\mathclose\bgroup\originalleft}
\renewcommand{\right}{\aftergroup\egroup\originalright}
%%

\def\code#1{\texttt{#1}}

% Theorem enviroments ----------------------------------------------------------
% http://www.ctex.org/documents/packages/math/amsthdoc.pdf

%%
\newtheoremstyle{break}% name
  {}%         Space above, empty = `usual value'
  {}%         Space below
  {}% Body font
  {}%         Indent amount (empty = no indent, \parindent = para indent)
  {\bfseries}% Thm head font
  {.}%        Punctuation after thm head
  {\newline}% Space after thm head: \newline = linebreak
  {}%         Thm head spec

\theoremstyle{break}
\newtheorem{thm}{Theorem}[chapter]   % Theorems, numbered by chapter
\newtheorem{lem}[thm]{Lemma}         % Lemmas, numbered like theorems
\newtheorem{prop}[thm]{Proposition}  % Propositions, numbered like theorems
\newtheorem{cor}[thm]{Corollary}         % Corollaries, numbered like theorems

\newtheorem{defn}[thm]{Definition}   % Definitions, numbered like theorems
\newtheorem{exmp}[thm]{Example}      % Examples, numbered like theorems

\newtheorem*{remark}{Remark}         % Remarks, unnumbered
%%

%\theoremstyle{plain}                 % Bold title, italic body text

%\theoremstyle{definition}            % Bold title, upright body text
%\newtheorem{thm}{Theorem}[chapter]   % Theorems, numbered by chapter
%\newtheorem{lem}[thm]{Lemma}         % Lemmas, numbered like theorems
%\newtheorem{prop}[thm]{Proposition}  % Propositions, numbered like theorems
%\newtheorem*{cor}{Corollary}         % Corollaries, unnumbered

%\newtheorem{defn}[thm]{Definition}   % Definitions, numbered like theorems
%\newtheorem{exmp}[thm]{Example}      % Examples, numbered like theorems

%\newtheorem*{remark}{Remark}

% Figure commands --------------------------------------------------------------

% imgfig ("image figure")
% Shortcut command for inserting an image from the fig/img folder
% Arguments:
%   * (optional) figure width; percentage of text width (default: 0.75)
%   * file name (without fig/img/ and file extension); also used for label
%   * the figure caption
% Examples:
%   \imgfig{image-name}{Caption goes here}
%   \imgfig[0.5]{image-name}{Caption goes here}
\newcommand{\imgfig}[3][0.75]{
  \begin{figure}[htbp]
    \centering
    \includegraphics[width=#1\textwidth]{fig/img/#2}
    \caption{#3}
    \label{fig:#2}
  \end{figure}
}

% dimgfig ("double image figure")
% Shortcut command for inserting two images side by side
% Arguments:
%   * (optional) width split ratio (default: 0.5, i.e. even split)
%   * file name for the left figure, without fig/img/ and file extension
%   * caption for the left figure
%   * file name for the right figure, without fig/img/ and file extension
%   * caption for the right figure
% Examples:
%   \dimgfig{img1}{First caption}{img2}{Second caption}
%   \dimgfig[0.3]{img1}{First caption}{img2}{Second caption}
% Alteratively, see
% https://en.wikibooks.org/wiki/LaTeX/Floats,_Figures_and_Captions#Subfloats
\newcommand{\dimgfig}[5][0.5]{
  \ifx\dimgleftwidth\undefined
    \newlength{\dimgleftwidth}
    \newlength{\dimgrightwidth}
  \fi
  \setlength{\dimgleftwidth}{#1\textwidth-0.02\textwidth}
  \setlength{\dimgrightwidth}{0.96\textwidth-\dimgleftwidth}
  \begin{figure}[htbp]
    \centering
    \begin{minipage}[t]{\dimgleftwidth}
      \centering
      \includegraphics[width=\linewidth]{fig/img/#2}
      \caption{#3}
      \label{fig:#2}
    \end{minipage}
    \hfill
    \begin{minipage}[t]{\dimgrightwidth}
      \centering
      \includegraphics[width=\linewidth]{fig/img/#4}
      \caption{#5}
      \label{fig:#4}
    \end{minipage}
  \end{figure}
}